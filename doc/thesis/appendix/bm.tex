\chapter{Battery Monitor for AAUSHIP}
\label{ch:bm}

\head{This appendix is a minor chapter describing a hardware
	component, the battery monitor, for AAUSHIP which was also devised
	under the project period. This is not directly a goal as according
	the project thesis, hence this is included in the appendix for
	reference.}

Since the AAUSHIP is equipped with multiple LiPo cells, it is of
a high concern to have a means of checking the state of these cells.
Therefore a battery monitor was created that can be hooked up on the
\ac{LLI} via the I$^2$C bus. This should enable the \ac{HLI} to report
the voltages on the cells. This was implemented with \ac{ROS} by
utilizing the graphical interface for \ac{ROS} called rqt. In this Qt
based environment it is possible to make plugins as easy as it is to
create \ac{ROS} nodes. A screenshot of the plugin in actions is seen
on figure~\vref{fig:bm-rqt}.

The battery monitor is made of two \ac{ADC} chips, namely the MCP3428,
each with four channels and 16-bit resolution. But not all this
resolution is of real use of such, because it is implemented with a
resistor voltage divider from each cell in series, which basically
means that only the top end of the value ranges are used. 

It is designed to measure maximum 4.2 V per cell to minimum of about
3.6 V. The four cells are in series, each with a resistor divider with
reference to the negative lead. This means that the voltage on high
cell is ranging from about 1.728 V to 2 V. In turn meaning that the
reading will only use the top  13.28 \% of the capabilities of the 
\ac{ADC}. This results in a resolution of 

\begin{figure}[H]
	\centering
	\includegraphics[width=0.8\textwidth]{fig/bm-macro}
	\caption{Macro photo of the finished battery monitor.}
	\label{fig:bm-macro}
\end{figure}

\begin{figure}[H]
	\centering
	\includegraphics[width=\textwidth]{fig/bm-rqt}
	\caption{Screen shot of the battery monitor plugin in rqt.}
	\label{fig:bm-rqt}
\end{figure}

\includepdf{pdf/bm-pcb}
\includepdf[angle=90]{pdf/bm-sch}
