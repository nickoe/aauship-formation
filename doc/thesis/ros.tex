\chapter{\acs{ROS} design}
\head{This chapter describes the details of the implementation level,
which is done on a \acl{ROS} platform. This should provide all the
information needed to complete the implementation.}

It was decided to use \ac{ROS} in the implementation, It is used as
an other abstraction layer on top of the \ac{LLI}. In turn making
AAUSHIP more modular and make it easy for others to write parts of the
control system without reimplementing everything. \todo{Det kunne være
	smart at få ordet ``extensible'' in her}

\ac{ROS} is a project available at \url{http://ros.org}, which
describes itself in short as following:

\begin{quote}
\noindent	\textit{
	ROS (Robot Operating System) provides libraries and
	tools to help software developers create robot applications. It
	provides hardware abstraction, device drivers, libraries,
	visualizers, message-passing, package management, and more. ROS is
licensed under an open source, BSD license.}
		
	\hfill ROS.org
\end{quote}

\section{\acs{ROS} terminology}
To start working with \ac{ROS} it is important to use the terminology
used by \ac{ROS} to avoid confusion. Therefore these  will be stated
in this section.  The idea of \ac{ROS} is to make it easy to build a
system modularly, and this is achieved byt using almost
``self-contained'' code segments called \textit{nodes}, which is
application parts that is run as its own process. A node should be
designed to execute limited tasks such as image processing or similar
atomic processes. These nodes can then communicate with other nodes by
the means of two main communication forms called \textit{topics} and
\textit{services}.

\begin{description}
\item[The topic] is an asynchronous connection, that can \textit{publish}
from many nodes and be \textit{subscribed} by many nodes.
\item[The service] is a synchronous connection that is used between one node
to another node.
\end{description}
\todo{Explaing tocic and service models better, keyowrd is unicast,
multicast, asynchronous and synchronous}

\begin{figure}[htbp]
	\centering
	\includesvg{ros_node_simple_concept}
	\caption{Basic principle of the node abstraction illustrating a
	service and two topics. The topology choosen here is only to illustrate
	the possibilities.}
	\label{fig:ros-node-simple-concept}
\end{figure}

When that is said, that is not the whole picture of the topology. In a
need to make this flexible \ac{ROS} has made it such that the nodes
can be started and stopped kind of ``runtime''. That is such that it is
possible to have different configurations of nodes to run in different
scenarios, i.e. in development with debugging nodes and virtual sensor
nodes versus in the real mission where no debugging nodes is used and
real sesnor nodes that use real sensor data is used.

\begin{figure}[htbp]
	\centering
	\includesvg{ros_node_master_concept}
	\caption{Concept showing the ROS master together with the nodes,
	also illustrating the masters role with multiple machines. Dashed
	lines hows that the node will either subscribe or publish to the
	topic. This only happens initially when connecting to a topic. Gray
	area is two physical separate machines.}
	\label{fig:ros-node-master-concept}
\end{figure}

\todo{Forklar noget om node håndtering, og master og storken og
blomsterne og bierne, og illustrationer er godt for den som lærer ved
at se det.}

