\chapter{\acs{ROS} design}
\head{This chapter describes the details of the implementation level,
which is done on a \acl{ROS} platform. This should provide all the
information, needed to complete the implementation.}

It was decided to try to use \ac{ROS} in the implementation, in the
hope to make the authors familiar with this platform, and use this as
an other abstraction layer on top of the \ac{LLI}. In turn making
AAUSHIP more modular and make it easy for others to write parts of the
control system without reimplementing everything. \todo{Det kunne være
	smart at få ordet ``extensible'' in her}

\ac{ROS} is a project available at \url{http://ros.org}, which
describes itself in short as following:

\begin{quote}
\noindent	\textit{
	ROS (Robot Operating System) provides libraries and
	tools to help software developers create robot applications. It
	provides hardware abstraction, device drivers, libraries,
	visualizers, message-passing, package management, and more. ROS is
licensed under an open source, BSD license.}
		
	\hfill ROS.org
\end{quote}



\section{\acs{ROS} terminology}
To start working with \ac{ROS} it is important to use the terminology
used by \ac{ROS} and therefore these  will be stated in this section.
The idea of ros is to make it easy to build a system very modularly,
and this is achieved byt using almost ``self-contained'' blocks called
\textit{nodes}, which is application parts that is run as its own
process. A node should be designed to execute limited tasks such as
image processing or similar atomic processes. These nodes can then
communicate with other nodes by the means of two main communication
forms called \textit{topics} and \textit{services}.

The topic is an asynchronous connection, that can ``publish'' from many
nodes and be ``subscribed'' by many nodes.

The service is a synchronous connection that is used between one node
to another node.

\todo{Forklar noget om node håndtering, og master og storken og
	blomsterne og bierne, og illustrationer er godt for den som lærer
ved at se det.}
