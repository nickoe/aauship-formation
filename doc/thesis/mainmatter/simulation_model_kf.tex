\section{Kalman Filter design}
\label{sec:kfdesign}

A \ac{KF} is one type of observer that can be applied to estimate the state vector. This is done to filter the measurements from the vessel and smooth these. If the measurements are too noisy, such that the vessel changes direction suddenly, but should be surging forward, a filter can predict the modelled direction and compare this to the measurement. This can be implemented as a low pass filter to make the changes in the measurements smaller with respect to the model predictions.

The \ac{KF} comprises the deterministic part of the model which estimates the state vector. This is corrected by means of measurements to estimate the final state vector. The process is modelled as seen on figure ~\ref{fig:blockkf} where the \ac{KF} is included.

\begin{figure}
	\centering
	\includesvg{kf_on_sys}
	\caption{Block diagram of kalman filter estimating the state vector.}
	\label{fig:blockkf}
\end{figure}

Here does the upper part of the figure represent the process, which can both be a simulated model or the real vessel with measurements. The lower part is the \ac{KF} which takes in the measurements and estimates the new state vector based on these measurements. The \ac{KF} can be of different types: \ac{LKF}, \ac{EKF} or \ac{UKF}. The choosing between these types of filters depends on the type of model used and the application. 

The \ac{LKF} prediction and update can be written as:
\begin{itemize}
\item Prediction
\begin{align}
\hat x_k^- &= \Phi\ \hat x_{k-1}^- + G u_k \\
P_k^- &= \Phi\ P_{k-1}^- \Phi^T + Q_{k}
\end{align}

\item Update
\begin{align}
\tilde y_k &= z_k - H_k\ \hat x_k^-\\
S_k &= H_k\ P_k^-H_k^T + R_k\\
K_k &= P_k^-H_k^TS_k^-\\
\hat x_k^+ &= x_k^- + K_k \tilde y_k\\
P_k^+ &= (I - K_k H_k) P_k^-
\end{align}
\end{itemize}
%\begin{itemize}
%\item Prediction
%\begin{align}
%\hat x_{k}^- &= \Phi_{k}\ \hat x_{k-1}^- + G_{k}\ u_{k-1}\nonumber\\
%\hat z_{k}^- &= H_{k}\ \hat x_{k}^-\nonumber\\
%P_{k}^- &= \Phi_{k}\ P_{k-1}^-\ \Phi_{k}^T + Q_{k}\nonumber
%\end{align}
%
%\item Update
%\begin{align}
%K_{k} &= P_{k}^- H_{k}^T(H_{k} P_{k}^- H_{k}^T+R_{k})^{-1}\nonumber\\
%\hat x_{k}^+ &= \hat x_{k}^- + K_{k}(z_{k}-\hat z_{k}^-)\nonumber\\
%P_{k}^+ &= (I-K_{k}\ H_{k})P_{k}^-\nonumber
%\end{align}
%\end{itemize}
where $P_{k}^-$ is the covariance propagation, $P_{k}^+$ is the update of covariance propagation, $Q$ is a covariance matrix with sensor variances and $R$ is a covariance matrix with model variances. $Q$ is a measure of how much the model is to be trusted. If the variance of the sensors are high this will imply that the model are to be trusted more than the noisy sensor measurements. These variances can sometimes be measured directly at the sensors and used in the $Q$ matrix. This leaves the $R$ matrix as the only design matrix left. $R$ is a measure of how much the measurement are to be trusted. If the variance of the model are high it might be better to trust the actual measurements. $z_k$ is the measurements from the sensors and $y_k$ is the difference between the measurements and the predicted stave vector, $\hat x_k^-$. $S_k$ is the covariance matrix of the residual with the variance of the model included. $K_k$ is the optimal Kalman gain, in a MMSE sense. 

The above mentioned \ac{KF} can only be applied on linear systems and transitions. Therefore is this not suited at the AAUSHIP. The position from the \ac{GPS} and the acceleration measurements needs to be rotated with a rotational matrix, which leads to non-linearities in the system. This can be seen on figure \vref{fig:intermediate-calc}. This entails that a \ac{LKF} cannot be used and an \ac{EKF} can be suited. The \ac{EKF} is used to linearise the non-linear terms in the system around the current estimate. In this case it will linearise the transition around the current measurements from the sensors to estimate the true output. The \ac{EKF} can be formulated in discrete form with the prediction and an update as:
\begin{itemize}
\item Prediction
\begin{align}
\hat x_k^- &= f(\hat x_{k-1}^-,u_{k-1})\\
P_k^- &= F_{k-1}P_{k-1}^-F_{k-1}^T+Q_{k-1}
\end{align}

\item Update
\begin{align}
\tilde y_k &= z_k - h(\hat x_k^-)\\
S_k &= H_k\ P_k^-H_k^T + R_k\\
K_k &= P_k^-H_k^TS_k^-\\
\hat x_k^+ &= x_k^- + K_k \tilde y_k\\
P_k^+ &= (I - K_k H_k) P_k^-
\end{align}
\end{itemize}
where the state transition and observation matrices are defined by their respective jacobians:
\begin{align}
F_{k-1} &= \left.\frac{\delta f}{\delta x}\right|_{\hat x_{k-1}^-,u_{k-1}}\\
H_k &= \left.\frac{\delta h}{\delta x}\right|_{\hat x_{k}^-}
\end{align}

The input is given by the forces applied to the vessel:
\begin{align}
u(k) = \tau =
\begin{bmatrix}
X\\
Y\\
0\\
0\\
N
\end{bmatrix}
\end{align}
These are the forces that can be applied by the thrusters mounted at the vessel. It should be noted that the pitch and roll are not set as input, since no thrusters can control these, and should be treated as model variations.

\begin{align}
H_k =
\begin{bmatrix}
1 & 0 & 0 & 0 & 0 & 0 & 0 & 0 & 0 & 0 \\
0 & 1 & 0 & 0 & 0 & 0 & 0 & 0 & 0 & 0 \\
0 & 0 & 0 & 0 & 1 & 0 & 0 & 0 & 0 & 0 \\
0 & 0 & 0 & 0 & 0 & 1 & 0 & 0 & 0 & 0 \\
0 & 0 & 0 & 0 & 0 & 0 & 1 & 0 & 0 & 0 \\
0 & 0 & 0 & 0 & 0 & \frac{\delta}{\delta u} & 0 & 0 & 0 & 0 \\
0 & 0 & 0 & 0 & 0 & 0 & \frac{\delta}{\delta v} & 0 & 0 & 0
\end{bmatrix}
\end{align}

\begin{align}
G = \text{The 10-by-5 matrix from the discrete model, named '$Bd$'}
\label{eq:gmatrix}
\end{align}
The matrix from \ref{eq:gmatrix} takes the forces in $X$, $Y$ and $N$ as input, and neglects inputs in $\phi$ and $\theta$. The forces in $\phi$ and $\theta$ are outputs from the system that makes the vessel change in pitch and roll. It is the same matrix used in the discrete model of the vessel, from the standard state space representation:
\begin{align}
\dot x = Ax + Bu
\end{align}
where the $G$ matrix is represented as the $B$ matrix.

The dimensions of the different matrices used are checked as a type of sanity check and verify that the matrices are in correct size.

The covariance propagation matrix through the system, the uncertainty of the estimated state, is given by:
\begin{align}
P_k^- &= F_k\ P_{k-1}^-F^T + Q_k\\
\text{dim}(P_k^-) &= [10x10]\cdot [10x10]\cdot [10x10]^T + [10x10]
\end{align}


The posteriori error covariance matrix, the update, is given by:
\begin{align}
P_k^+ &= (I - K_k\ H_k)P_k^-\\
\text{dim}(P_k^+) &= ([10x10] - [10x7]\cdot [7x10])\cdot [10x10]
\end{align}
which is a measure of the estimated accuracy of the state estimate.

Adding a middle calculation as the residual covariance:
\begin{align}
S_k &= H\ P\ H^T + R\\
\text{dim}(S_k) &= [7x10]\cdot [10x10]\cdot [7x10]^T + [10x10]
\end{align}

The updated Kalman Gain:
\begin{align}
K_k &= P\ H^T\ S^{-1}\\
\text{dim}(K_k) &= [10x10]\cdot [7x10]^T\cdot [7x7]^{-1}
\end{align}
witch is optimal in a MMSE sense.