\subsection{Determine R}
The matrix R represents covariance matrix of observational (measurement) uncertainty. This is the matrix which sets the individual variances of the specific measurements from the sensors, the measurements from the output vector $z$. This means that the coefficients of the R matrix can be determined within some interval. The disturbances of each sensor measurement highly depends on which sensor is implemented in the used system. When looking at the AAUSHIP the sensors are of higher accuracy, which means that the sensors are to some extend trustworthy. There are two types of sensors in the AAUSHIP, one \ac{GPS} and an \ac{IMU}. There are two types of \ac{GPS}s installed; a \ac{RTK} \ac{GPS} and a standard \ac{GPS}. \todo{Her kan kun naevnes dists from den alm pt - Skriv noget mere efter snak med Nick}. The \ac{IMU} consists of a magnetometer, a gyro and an accelerometer. The magnetometer has a measured accuracy of \todo{Indsaet og forklar hurtigt}. The gyro has a accuracy of \todo{Indsaet for x y z og forklar hurtigt}. The accelerometer has a measured accuracy of \todo{Indsaet for x y z og forklar hurtigt}. The \ac{GPS} have different accuracies dependent on which type is used. The main type used will be the \ac{RTK} \ac{GPS} which has higher accuracy than the standard \ac{GPS}. The resulting R matrix is measured at the measurements of $z$ and determined to be
\begin{align}
&R_k = \diag{\mathbf{v}} =\\ \nonumber
&\diag{^\text{GPS}\sigma_{N}^2,\ ^\text{GPS}\sigma_{E}^2,\ ^\text{GPS}\sigma_{\psi}^2,\ ^\text{GPS}\sigma_{u}^2,\ ^\text{GPS}\sigma_{v}^2,\ ^\text{IMU}\sigma_{a_x}^2,\ ^\text{IMU}\sigma_{a_y}^2}
\end{align}


\subsection{Determine Q}
The matrix Q represents the covariance matrix of process noise in the system state dynamics. This matrix includes parameters as disturbances in the process itself. When looking at the AAUSHIP this could for instance be the incoming waves acting as disturbances both in the roll, pitch and the heading. These types of disturbances can be hard to measure and put a precise number to. Instead this matrix is used as a tuning parameter for the \ac{KF} where this sets a weighting of how much the process, and therefore the model, are to be trusted. If the sensors has a high accuracy, it would be of benefit to trust these more than the process, thus putting the parameter of this specific measurement in the Q matrix high. But the Q matrix is used to combine the measurements and the model prediction together. This means that based on the variances from the Q matrix, based on the values from the R matrix, is tuned to get the best performance from the combination of the model and the measurements.\\
As the model of the AAUSHIP is made makes it possible to tune the parameters of Q in a systematic way. The model decouples the acceleration from the velocity, and the velocity from the position. Therefore it is of benefit to tune the acceleration noise variance firstly, such that this makes a proper fit to a step function. A simple test can be simulated where the AAUSHIP accelerates to a certain velocity and keeps this velocity. This needs to fit within some predetermined margin that makes the AAUSHIP follow a satisfying curve in each of the acceleration, velocity and position. The simulation curve is both based on measurements and model predictions \todo{Vi skal have sat nogle tal på det her, maaske forme en krav spec? - Dog er det jo mest the relative position til trajectory som har en vigtigere krav}. A wanted acceleration curve of the acceleration, for instance in surge, will look like \todo{Få lige sat billede ind}. Afterwards is the velocity of the AAUSHIP tuned. This is done in the same manor, where the velocity of a step function needs to look like \todo{Også billede her}. At last is the position tuned, which is the one with largest relative noise from the measurements. This forces the value in Q to be smaller thus making the AAUSHIP trust the model prediction more than then measurements. A position plot with two different values in Q can be seen here. \todo{lav plot med to forskellige Q værdier i N og E}. The one with the lowest value in Q gives the best position fit of these, which also follows the intuition of how the \ac{KF} should work. The resulting Q matrix is tuned to be
\begin{align}
&Q_k = \diag{\mathbf{w}} = \\ \nonumber
&\diag{\sigma_{N}^2, \sigma_{E}^2, \sigma_{x_{b}}^2, \sigma_{y_{b}}^2, \sigma_{\phi}^2, \sigma_{\theta}^2, \sigma_{\psi}^2, \sigma_{u}^2, \sigma_{v}^2, \sigma_{p}^2, \sigma_{q}^2, \sigma_{r}^2, \sigma_{\dot u}^2, \sigma_{\dot v}^2, \sigma_{\dot p}^2, \sigma_{\dot q}^2, \sigma_{\dot r}^2}
\end{align}


\subsection{When absence of \ac{GPS} signal}
When there is no \ac{GPS} signal it is not possible to make any new estimates based from the measurements. This results in a model based phase of the controlling of the AAUSHIP. Therefore the AAUSHIP needs to converge to the predetermined trajectory based on how the model of the ship would do it. When a new \ac{GPS} signal is present this needs to be taken into account and thereby used as a correction to the model prediction. When the signal is absent there are different ways to handle this in the model. One of the ways is to set the variance of the sensor noise from the \ac{GPS} high. This value could be around $10 \cdot 10^9$ to ensure that the model does not take the measurement from the \ac{GPS} into account.

\subsection{Sample rates}
If the different sensor measurements are not sampled with the same rate will the lower sample rates be untrue. This induces the same situation as if the \ac{GPS} signal is absent, thus setting the variances of these measurements high. The same situation will occur as with the missing \ac{GPS} signal, making the measurement be highly untrusted and only taking in the model prediction for that particular sample.

\subsection{Overall filter test and conclusion}
Ja det er nok meget godt.















"Det lange af det korte er" - LOL, sagt af en buisness mand i toget, han meste lidt det modsatte. Selvom samtalen faktisk blev rimelig lang :p #FolkFraÅrhus