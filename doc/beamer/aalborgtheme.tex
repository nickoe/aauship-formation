\documentclass[10pt,handout]{beamer}
\usetheme[
%%% options passed to the outer theme
%    hidetitle,           % hide the (short) title in the sidebar
%    hideauthor,          % hide the (short) author in the sidebar
%    hideinstitute,       % hide the (short) institute in the bottom of the sidebar
%    shownavsym,          % show the navigation symbols
%    width=2cm,           % width of the sidebar (default is 2 cm)
%    hideothersubsections,% hide all subsections but the subsections in the current section
%    hideallsubsections,  % hide all subsections
%    right                % right of left position of sidebar (default is right)
  ]{Aalborg}

\definecolor{aaublue}{RGB}{33,26,82}
\definecolor{aaugrey}{RGB}{84,97,110}

% If you want to change the colors of the various elements in the theme, edit and uncomment the following lines
% Change the bar and sidebar colors:
\setbeamercolor{Aalborg}{fg=aaublue!10,bg=aaugrey!60}
%\setbeamercolor{sidebar}{bg=blue!74}
% Change the color of the structural elements:
\setbeamercolor{structure}{fg=aaublue}
 \setbeamercolor{subtitle}{fg=aaugrey}
% Change the frame title text color:
\setbeamercolor{frametitle}{fg=aaublue}
% Change the normal text color background:
%\setbeamercolor{normal text}{bg=aaugrey!10}
% ... and you can of course change a lot more - see the beamer user manual.
\usebackgroundtemplate{\includegraphics[width=\paperwidth]{img/background}}

\usepackage[utf8]{inputenc}
\usepackage[english]{babel}
\usepackage[T1]{fontenc}
\usepackage{subfig} % Vi kan nu bruge \subfloat
\usepackage{soul} % use this (many fancier options)
\DeclareUnicodeCharacter{00A0}{~} % Fixes the "! Package inputenc Error: Unicode char \u8:  not set up for use with LaTeX."
% ... or whatever. Note that the encoding and the font should match. If T1
% does not look nice, try deleting the line with the fontenc.
\usepackage{lmodern} %optional

\usepackage[svgpath=img/]{svg} % Smart inclusion of SVG figures with \includesvg

% colored hyperlinks
\newcommand{\chref}[2]{%
  \href{#1}{{\usebeamercolor[bg]{Aalborg}#2}}
}

\title[AAUSHIP\\ Formation Control]% optional, use only with long paper titles
{\includegraphics[width=5cm]{../thesis/frontmatter/aauship}\\ Halfway Statusseminar for the\\ AAUSHIP Formation Control Project}

%\subtitle[v.\ 0.1.1] %optional
%{v.\ 0.1.1}

\author[14gr1034]{% optionally input the group number, use only with lots of authors
  Nick Østergaard \and Jeppe Dam\\
  {{\tt \{nickoe,jeppedam\}@es.aau.dk}}
}
% - Give the names in the same order as they appear in the paper.
% - Use the \inst{?} command only if the authors have different
%   affiliation. See the beamer manual for an example

%specify some optional logos
\pgfdeclareimage[height=1.2cm]{mainlogo}{aau_logo.pdf} % placed in the upper left/right corner
\logo{\pgfuseimage{mainlogo}}

\pgfdeclareimage[height=0.75cm]{logo2}{tu-logo} % placed in the lower left/right corner if the \pgfuseimage{logo2} command is uncommented in the \institute command below

\institute[
%  {\pgfuseimage{logo2}}\\ %insert a company or department logo
  Dept.\ of Electronic Systems,\\
  Aalborg University,\\
  Denmark
] % optional - is placed in the bottom of the sidebar on every slide
{%
  Department of Electronic Systems,\\
  Aalborg University,\\
  Denmark
  
  %there must be an empty line above this line - otherwise some unwanted space is added between the university and the country (I do not know why;( )
}
\date{\today}

\begin{document}
\setbeamertemplate{caption}{\raggedright\insertcaption\par}
% the titlepage
\begin{frame}[plain] % the plain option removes the sidebar and header from the title page
  \titlepage
\end{frame}
%%%%%%%%%%%%%%%%

% TOC
\begin{frame}{Agenda}{}
\tableofcontents
\end{frame}
%%%%%%%%%%%%%%%%
\section{Introduction}
\begin{frame}{Introduction}{Motivation}
  \begin{itemize}
    \item Port of Aalborg wants to be an intelligent harbour
    \item Always updated bathymetry data for an ``automatic pilot''
    \item Currently mapped with a manned boat
    \begin{itemize}
      \item Previosly using a single beam echoshounder
      \item Now using a multibeam sonar
    \end{itemize}
    \item Bathymetry data is currently updated every three month to every three years approximately
    \item Could use autonomous vehichles to verify validity of old data
    \item A fleet of ``cheap'' autonomous vehichles could update the map in areas which do not require high resolution.
  \end{itemize}
\end{frame}

\begin{frame}{Introduction}{Motivation}
  \begin{figure}
	  \includegraphics[width=\textwidth]{../thesis/fig/use-case-data}
	  \caption{\scriptsize Area of the harbour at Aalborg Portland provided as sample
	  data from Aalborg Havn. Background map data CC BY-SA OpenStreetMap.}
  \end{figure}
\end{frame}

\begin{frame}{Introduction}{The Mission}
\begin{itemize}
\item Use Autonomous Surface Vessel's (ASV's) for mapping with a single beam sonar
\item Control the ASV's with some kind of formation control
\item Formation control is choosen because it is an advantage to only have one group
\end{itemize}
\end{frame}

\begin{frame}{Introduction}{Concept Art}
\begin{figure}
  \centering
  \subfloat[One ship\label{fig:concept-art1}]{\includegraphics[width=0.48\textwidth]{../thesis/fig/conseptart-single}}
  \ % One forced space to seperate figures
  \subfloat[Thee ships\label{fig:concept-art3}]{\includegraphics[width=0.48\textwidth]{../thesis/fig/conseptart-formation}}
  \caption{Comparison of two ways to cover an area with a single beam ships}
  \label{fig:concept-art}
\end{figure}
\end{frame}

\begin{frame}{Introduction}{1st Half Goals}
Development of a single working testing platform
  \begin{itemize}
  \item Research of formation control paradigms
  \item Decide feasibility for using ROS as the software platform
  \item Subsystem test of the ship system, manual control
  \item Modelling of the ship
  \item Design and implementation of an state estimator
  \item Creation of a simulation environment (Matlab and ROS)
  \item \textit{Testing of autonomous control for a single ship}
  \item \textit{Further research for the formation control task}
  \end{itemize}
\end{frame}


%%%%%%%%%%%%%%%%
\section{System Design}
\begin{frame}{System Design}{AAUSHIP.01}
\begin{itemize}
  \item<1-> The ship is designed as a non-planing deplacement craft (eg. like freight ships).
  \item<2-> Developed using rapid prototyping techniques.
  \item<3-> Developed in Rhinoceros using a lofting techniques.
  \item<4-> Printed on a 3D printer.
  \item<5-> Examined and the process iterated.
  \item<6-> Vaccumformed by DD-plast in Randers and assembled in the machine shop at Aalborg University.
\end{itemize}
<Image of the boat here>
\end{frame}

\begin{frame}{System Design}{Hardware}
\begin{itemize}
  \item Fitted with 2 x 1200W engines (totally producing around 3 HP at full thrust).
  \item Fitted with 6 x 3200mAh batteries (results in a mission time of around 5 hours).
  \item 2 counter rotating 60mm propellers.
  \item Inertial Measurement Unit.
  \item Global Positioning System.
  \item A (currently) 20mW 19.2 kbps radio link @ 470 MHz
  \item Arduino Mega with a custom made shield board mounted.
  \item Retrofitted with a hydrofoil to reduce the wake and pitch of the ship.
\end{itemize}
\end{frame}

\begin{frame}{System Design}{Overview}
Modelling
\begin{itemize}
% <Describe what should be modelled, why 5DOF, parmeter estimation...>
\item What should be modelled?
\item How many states
\item Parameter estimation
\end{itemize}

Estimaiton
\begin{itemize}
% <KF and AHRS filters>
\item KF
\item AHRS
\end{itemize}

Simulation
\begin{itemize}
% <Simulation to model verification>
\item Simulation model in Matlab and ROS for office testing
\end{itemize}

Implementation with ROS
\begin{itemize}
\item ROS is modular, theese are called nodes
\end{itemize}
\end{frame}

\begin{frame}{System Design}{ROS}
  \begin{figure}
    {\tiny \includesvg[width=\textwidth]{ros_aauship_closed_loop_single_beamer}}
	  \caption{\scriptsize The current ROS node and topic diagram, showing the elements that is implemented in black. Gray is to make it more user friendly.}
  \end{figure}
\end{frame}

%%%%%%%%%%%%%%%%
\section{Modelling}
\begin{frame}{Modelling}{Linear Model}

\end{frame}

\begin{frame}{Modelling}{Simulation Model}

\end{frame}

%%%%%%%%%%%%%%%%
\section{Simulation and Verificaiton}
\begin{frame}{Simulation and Verificaiton}{Kalman Filter}
Acceleration tuning
  \begin{figure}
    \includegraphics[width=0.9\textwidth]{../../code/matlab/accel0,00001}
    \caption{\scriptsize Tuned acceleration with the ending Q matrix.}
    \label{fig:acceltuning}
  \end{figure}
\end{frame}

\begin{frame}{Simulation and Verificaiton}{Kalman Filter}
Velocity tuning
  \begin{figure}
    \includegraphics[width=0.9\textwidth]{../../code/matlab/uv0,00001}
    \caption{\scriptsize Tuned velocity with the ending Q matrix.}
    \label{fig:uvtuning}
  \end{figure}
\end{frame}

\begin{frame}{Simulation and Verificaiton}{Kalman Filter}
Position tuning, $Q_{1:2,1:2}$ = 0.001
  \begin{figure}
    \includegraphics[width=0.7\textwidth]{../../code/matlab/q0,001}
    \caption{\scriptsize Tuned position with 0.001 in Q matrix.}
    \label{fig:q0.001}
  \end{figure}
  The min/max error is 0.1m/2.2m
\end{frame}

\begin{frame}{Simulation and Verificaiton}{Kalman Filter}
Position tuning, $Q_{1:2,1:2}$ = 0.0001
  \begin{figure}
    \includegraphics[width=0.7\textwidth]{../../code/matlab/q0,0001}
    \caption{\scriptsize Tuned position with 0.0001 in Q matrix.}
    \label{fig:q0.0001}
  \end{figure}
  The min/max error is 0.1m/1.8m
\end{frame}

\begin{frame}{Simulation and Verificaiton}{Kalman Filter}
Position tuning, $Q_{1:2,1:2}$ = 0.00001
  \begin{figure}
    \includegraphics[width=0.7\textwidth]{../../code/matlab/q0,00001}
    \caption{\scriptsize Tuned position with 0.00001 in Q matrix.}
    \label{fig:q0.00001}
  \end{figure}
  The min/max error is 0.1m/1.6m
\end{frame}

\begin{frame}{Simulation and Verificaiton}{Attitude and Heading Reference System}
<plots>
\end{frame}

\begin{frame}{Simulation and Verificaiton}{Comparision between KF and AHRS}
  \begin{figure}
    \includegraphics[width=0.9\textwidth]{../../code/matlab/mahonyvskf}
    \caption{\scriptsize Heading tracking by the KF and the Mahony filters.}
    \label{fig:mavskf}
  \end{figure}
\end{frame}

\begin{frame}{Simulation and Verificaiton}{ROS Demo}
<video>
\end{frame}

\begin{frame}{Simulation and Verificaiton}{Loose Ends}
\begin{itemize}
  \item Interface between the \texttt{/lli\_input} topic is not compatible with the LLI
  \item Optionally extend the KF with the the second GPS and use dillution of precision to modify the observational covariance matrix
  \item Thrustallocation needs testing
  \item Lab test with the hardware in the loop to test that interfaces work
  \item Sea-trail needed to verify that it all works
\end{itemize}
\end{frame}


%%%%%%%%%%%%%%%%
\begin{frame}{Testing the control algorithms}{Results}
  \begin{itemize}
  \item Plot of the ship states during the voyage
  \end{itemize}
  \begin{figure}
    \begin{center}
      \includegraphics[width=8.2cm]{img/states}
      \label{fig:controltest3}
    \end{center}
  \end{figure}
\end{frame}

%%%%%%%%%%%%%%%%
\section{Schedule}
\begin{frame}{Schedule}{2nd Half Goals}
Design and implementation of formation control on the platforms
  \begin{itemize}
  \item Assembly of two other boats, same as the we have
  \item Analyse different formation control strategies in more detail
  \item Determination of practical group coordination (initialisation task)
  \item Implementation of leader-follower formation control (tracking task)
  \item Formation control is according to a given path of interest
  \item Extend simulation model with environmental disturbances
  \item Agent-agent avoidance, and maybe other objects too
  \end{itemize}
\end{frame}

{
\setbeamertemplate{background canvas}{\centering\includegraphics[height=\paperheight,keepaspectratio]{{img/aauship}}}
\begin{frame}[plain]{}\end{frame}}

\end{document}
