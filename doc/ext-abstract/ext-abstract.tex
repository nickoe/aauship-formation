\documentclass[a4paper,12pt]{article}
\usepackage{natbib}
\usepackage{graphicx}
\usepackage[utf8]{inputenc} % So we can input Nicks name in the paper title!
\usepackage[T1]{fontenc}
\usepackage{amsmath,amsfonts,amssymb} % Added so we can do pretty math equations.
\usepackage{geometry}
\usepackage{lipsum}
\geometry{left=3cm,right=3cm,bottom=4cm}
\begin{document}

\title{\vspace{-2cm}Formation Control of AAUSHIP}
%\subtitle{Extended Abstract}
\author{Nick Østergaard \and Jeppe Dam}
\maketitle

%\begin{center}
%\vspace{-0.7cm}
%Group 12gr730
%\end{center}
%\thispagestyle{empty}

\paragraph{Background}
\lipsum[5]

\paragraph{Aim}
\lipsum[1]

\paragraph{Method}
\lipsum[2]

\paragraph{Results}
\lipsum[3]

\paragraph{Conclusion}
\lipsum[4]



\end{document}

%Here's a suggestion as to what an extended abstract should contain:
%
%    Background - A little history about who's done what and how your work fits in with it.
%    Aim - What you're trying to tell the audience that they don't already know (e.g. Your story.)
%    Method - Why the audience should believe that the results you've got aren't made up or flawed
%    Results - Evidence that you've come up with that confirms your story
%    Conclusion - Recap of your story and its implications
%    Limitations - Why someone might doubt your story and what you've done to get rid of as much doubt as possible.
%
%What if I'm presenting a review of my progress to date and I have no original research?
%Method = literature survey.
%Results = what you've read.
